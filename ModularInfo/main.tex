\documentclass{article}
\usepackage{amsmath}
\usepackage{amssymb}
\usepackage{amsthm}
\usepackage{amsfonts}
\usepackage[utf8]{inputenc}
\usepackage[]{graphicx}
\usepackage[a4paper, portrait, margin = 1in]{geometry}
\usepackage{enumitem}
\usepackage{xcolor}
\usepackage{newpxmath}
\usepackage{newpxtext}
\usepackage{scsnowman}

\DeclareMathSizes{10}{10}{7}{5} 

%\color[rgb]{0.2,0.19,0.18} 
%\pagecolor[rgb]{0.92,0.86,0.7}


\begin{document}
    \huge Modular Arithmetic Background
    \normalsize
\begin{enumerate} 

    \section{Introduction}

    To begin the background information, we start by defining modular congruence. We say that two numbers $a$ and $b$ are congruent modulo $n$ if their difference is divisible by $n$:
    \[
    a\equiv b\pmod{n}\iff n|a-b
    .\] 
    Modular congruence is an equivalence relation. This means it is:
    \begin{itemize}
        \item Reflexive: Every element of $\mathbb{Z}_n$ is equivalent to itself.
            \begin{proof} 
                Let $a\in \mathbb{Z}_n$. Then observe that $0=0n=(a-a)n$. So $n|a-a$ and $a\equiv a\pmod{p}$.
            \end{proof}
        \item Symmetric: If $a\equiv b\pmod{n}$, then $b\equiv a \pmod{ n   }$.
            \begin{proof} 
                
            \end{proof}
        \item Transitive:
    \end{itemize}

\end{enumerate}
\end{document}
