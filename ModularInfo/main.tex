\documentclass{article}
\usepackage{amsmath}
\usepackage{amssymb}
\usepackage{amsthm}
\usepackage{amsfonts}
\usepackage[utf8]{inputenc}
\usepackage[]{graphicx}
\usepackage[a4paper, portrait, margin = 1in]{geometry}
\usepackage{enumitem}
\usepackage{xcolor}
\usepackage{newpxmath}
\usepackage{newpxtext}
\usepackage{scsnowman}

\DeclareMathSizes{10}{10}{7}{5} 

%\color[rgb]{0.2,0.19,0.18} 
%\pagecolor[rgb]{0.92,0.86,0.7}


\begin{document}
    \huge Modular Arithmetic Background
    \normalsize
\begin{enumerate} 

    \section{Introduction}

    To begin the background information, we start by defining modular congruence. We say that two numbers $a$ and $b$ are congruent modulo $n$, or that their equivalence classes are equal, if their difference is divisible by $n$:
    \[
    a\equiv b\pmod{n}\iff n|a-b
    .\] 
    Recall that $a|b$ means that there is some integer $c$ so that $ac=b$.

    Modular congruence is an equivalence relation. This means it is:
    \begin{itemize}

        \item Reflexive: Every element of $\mathbb{Z}_n$ is equivalent to itself.
            \begin{proof} 
                Let $a\in \mathbb{Z}_n$. Then observe that $0=0n=(a-a)n$. So $n|a-a$ and $a\equiv a\pmod{p}$.
            \end{proof}
        \item Symmetric: If $a\equiv b\pmod{n}$, then $b\equiv a \pmod{n}$.
            \begin{proof} 
                Suppose $a\equiv b\pmod{n}$. Then there exists some $k\in \mathbb{Z}$
                so that $nk=a-b$. Then $n(-k)=b-a$. So $n|b-a$ and $b\equiv a\pmod{n}$.
            \end{proof}
        \item Transitive: If $a\equiv b\pmod{n}$ and $b\equiv c\pmod{n}$, then $a\equiv c \pmod{n}$.
            \begin{proof} 
                Suppose $a\equiv b\pmod{n}$ and $b\equiv c\pmod{n}$. Then we must have
                some $k,l\in \mathbb{Z}$ so that $kn=a-b$ and $ln=b-c$. Adding
                the second equality from the first, we get:
                \begin{align*}
                    kn- l n&= a-b+b-c \\
                    n(k- l) &= a-c 
                .\end{align*}
                And so we see that $a\equiv c\pmod{n}$.
            \end{proof}
    \end{itemize}
    
\end{enumerate}
\end{document}
