\documentclass{article}
\usepackage{amsmath}
\usepackage{amssymb}
\usepackage{amsthm}
\usepackage{amsfonts}
\usepackage[utf8]{inputenc}
\usepackage[]{graphicx}
\usepackage[a4paper, portrait, margin = 1in]{geometry}
\usepackage{enumitem}
\usepackage{xcolor}
\usepackage{newpxmath}
\usepackage{newpxtext}
\usepackage{scsnowman}

\DeclareMathSizes{10}{10}{7}{5} 

%\color[rgb]{0.2,0.19,0.18} 
%\pagecolor[rgb]{0.92,0.86,0.7}


\begin{document}
\huge Modular Arithmetic Background
\normalsize
\section{Introduction}

\subsection{Modular Congruence and its Properties}
To begin the background information, we start by defining modular congruence. We say that two numbers $a$ and $b$ are congruent modulo $n$, or that their equivalence classes are equal, if their difference is divisible by $n$:
\[
a\equiv b\pmod{n}\iff n|a-b
.\] 
Recall that $a|b$ means that there is some integer $c$ so that $ac=b$.

Modular congruence is an equivalence relation. This means it is:
\begin{itemize}

    \item Reflexive: Every element of $\mathbb{Z}_n$ is equivalent to itself.
        \begin{proof} 
            Let $a\in \mathbb{Z}_n$. Then observe that $0=0n=(a-a)n$. So $n|a-a$ and $a\equiv a\pmod{n}$.
        \end{proof}
    \item Symmetric: If $a\equiv b\pmod{n}$, then $b\equiv a \pmod{n}$.
        \begin{proof} 
            Suppose $a\equiv b\pmod{n}$. Then there exists some $k\in \mathbb{Z}$
            so that $nk=a-b$. Then $n(-k)=b-a$. So $n|b-a$ and $b\equiv a\pmod{n}$.
        \end{proof}
    \item Transitive: If $a\equiv b\pmod{n}$ and $b\equiv c\pmod{n}$, then $a\equiv c \pmod{n}$.
        \begin{proof} 
            Suppose $a\equiv b\pmod{n}$ and $b\equiv c\pmod{n}$. Then we must have
            some $k,l\in \mathbb{Z}$ so that $kn=a-b$ and $ln=b-c$. Adding
            the second equality from the first, we get:
            \begin{align*}
                kn- l n&= a-b+b-c \\
                n(k- l) &= a-c 
            .\end{align*}
            And so we see that $a\equiv c\pmod{n}$.
        \end{proof}
\end{itemize}

The consequence of this is that each element of $\mathbb{Z}_n$ is in exactly one equivalence class. In other words, $[a]_n=[b]_n,$ or $[a]_n\cap [b]_n=\varnothing$.

\subsection{Modular Arithmetic}

In order to do anything useful or interesting with modular arithmetic, we must be able to add and subtract within our expressions. We define these as follows:

\begin{itemize}
    \item If $a+b=c$, then $a+b\equiv c\pmod{n}$.
    \item If $ab=c$, then $ab\equiv c\pmod{n}$.
\end{itemize}
Thanks to these convenient definitions, we inherit convenient properties from the integers:
\begin{itemize}
    \item $a+0\equiv a\pmod{n}$, equivalently: $a-a\equiv 0\pmod{n}$
        \begin{proof} 
            $a=a+0$, so $a\equiv a+0\pmod{n} $.
        \end{proof}
    \item $1a\equiv a\pmod{n}$.
        \begin{proof} 
            $a=1a$, so $a\equiv 1a\pmod{n} $.
        \end{proof}
        It's important to note that we may not always have a multiplicative inverse, and when we do have one, it may not be unique.
\end{itemize}

\subsection{Inverses}

In the same way that we can find additive inverses, $a-a\equiv 0\pmod{ n} $, we seek to find multiplicative ones, so that $aa^{-1}\equiv 1\pmod{n} $.

It turns out that $a^{-1}$ exists if $\gcd(a,n)=1$. That is, $a$ and $n$ have no common factors.
\begin{proof} 
    Suppose $\gcd(a,n)=1$. By a helpful result called Bezout's lemma, we can say that $ax+ny=1$. Rearranging, we get $ny=ax-1$, and translating into modular language, we get $ax\equiv 1\pmod{n} $. So we have found a multiplicative inverse for $a\pmod{n} $.
\end{proof}

Can we count the number of invertible elements in $\mathbb{Z}_n$? Yes we can! It will make more sense why we do this later, but for now we simply introduce Euler's totient function $\varphi$. Define $\varphi(n)$ to be equal to the number of positive integers $d$ up to $n$ so that $\gcd(d,m)=1$.

Then since any prime $p$ is not divisible by any number less than itself (by definition of prime), we know that $\varphi(p)=p-1$.

Let's also take a look at numbers of the form $pq$, the product of two primes. Since prime factorizations are unique (See the fundamental theorem of arithmetic), we know that the only divisors of $pq$ are $p,q$. We can also show that $\varphi(pq)=(p-1)(q-1)$. This does generalize to further products of primes, in fact every number, but it quickly becomes complicated and the proof is longer than we'd like to include.

\subsection{Using the Totient}

The motivating theorem for RSA cryptography is the following:

For any modulus $n$, and any $a$ that is coprime with $n$, the identity $a^{\varphi(n)}\equiv 1\pmod{n} $ holds.

The proof again is more complicated, using facts from abstract algebra, namely Lagrange's Theorem. 

\end{document}
