\documentclass{article}
\usepackage{hyperref}
\usepackage{amsmath}
\usepackage{amssymb}
\usepackage{amsthm}
\usepackage{amsfonts}
\usepackage[utf8]{inputenc}
\usepackage[]{graphicx}
\usepackage[a4paper, portrait, margin = 1in]{geometry}
\usepackage{enumitem}
\usepackage{xcolor}
\usepackage{newpxmath}
\usepackage{newpxtext}
\usepackage{scsnowman}

\DeclareMathSizes{10}{10}{7}{5} 

%\color[rgb]{0.2,0.19,0.18} 
%\pagecolor[rgb]{0.92,0.86,0.7}



\begin{document}
    \huge The Procedure
    \normalsize
    \bigskip
%TO DO: CHANGE THE ID'S OF EACH TEXTBOX
    \begin{enumerate}
\item Bob chooses prime numbers $p,q$ so that $pq=n>c$, where $c$ is the size of the largest message they expect to encrypt. The $n$ is called the key length.

\item Bob also chooses $e,d$ by first computing $\varphi(n)=(p-1)(q-1)$, then choosing $e$ so that $\gcd(e,\varphi(n))=1$. Bob can then release the keypair $(e,n)$, and now Alice, Bob, and Eve respectively have their own step to perform.
    
    \paragraph{Bob:} Will compute the multiplicative inverse of $e\pmod{\varphi(n)}$, using Wolfram-Alpha, in order to find his private key $d$.

    \paragraph{Alice:} Will choose a message which is converted to a large natural number $M$. We have many ways of doing this, however for the purpose of this exercise we will use $a=1, b=2,\dots, z=26$, and a space is $00$. For messages longer than a single character, simply combine the letters. For example, a message "abc" will be encoded with $M=010203$. Alice encrypts her message text with Wolfram-Alpha using $M^e\pmod{n}$, and reveals it to both Bob and Eve.

    \paragraph{Eve:} Will use Wolfram-Alpha to attempt to factor $n$, and if successful, uses the $p,q$ she finds in order to find $d$ (see Bob's instructions above).

\item Now Bob and Eve (if successful) should reveal the decrypted messages; and confirm whether they recieved it.
\end{enumerate}

\newpage

    \huge Exercise 1
    \normalsize

    For the first exercise, delegate who will be Bob, Alice, and Eve, and perform the procedure with $p=2$ and $q=13$. This first exercise is exactly enough to encrypt a message of one character. As a group, fill out the following:
    \begin{enumerate}
        \item What was the message sent by Alice? 

            \TextField[width=6in,height=1in, bordercolor=0 0 0, name=p1q1, multiline=true]{}

        \item What was Bob able to decrypt?

            \TextField[width=6in,height=1in, bordercolor=0 0 0, name=p1q2, multiline=true]{}

        \item What was Eve able to encrypt?

            \TextField[width=6in,height=1in, bordercolor=0 0 0, name=p1q3, multiline=true]{}

        \item How private was the message?

            \TextField[width=6in,height=1in, bordercolor=0 0 0, name=p1q4, multiline=true]{}

        \item For what purposes would you use this level of encryption?

            \TextField[width=6in,height=1in, bordercolor=0 0 0, name=p1q5, multiline=true]{}

    \end{enumerate}

\newpage

    \huge Exercise 2
    \normalsize

    For the second exercise, use \verb|bigprimes.org| or some other random number generating site to choose some primes $p,q$ with more digits. Try to ensure that the primes are small enough that Eve has a chance at breaking the message. All participants will need Wolfram-Alpha.

    \begin{enumerate}
        \item What was the message sent by Alice? 

            \TextField[width=6in,height=1in, bordercolor=0 0 0, name=p2q1, multiline=true]{}

        \item What was Bob able to decrypt?

            \TextField[width=6in,height=1in, bordercolor=0 0 0, name=p2q2, multiline=true]{}

        \item What was Eve able to encrypt?

            \TextField[width=6in,height=1in, bordercolor=0 0 0, name=p2q3, multiline=true]{}

        \item How private was the message?

            \TextField[width=6in,height=1in, bordercolor=0 0 0, name=p2q4, multiline=true]{}

        \item For what purposes would you use this level of encryption?

            \TextField[width=6in,height=1in, bordercolor=0 0 0, name=p2q5, multiline=true]{}

    \end{enumerate}


    \newpage
    \huge Exercise 3
    \normalsize

    For the third and final exercise, use \verb|bigprimes.org| or some other random number generating site to choose some primes $p,q$ with enough digits that Bob and Alice can agree that Eve will not be able to factor $n$ using Wolfram-Alpha (How can they check this!). 

    \begin{enumerate}
        \item What was the message sent by Alice? 

            \TextField[width=6in,height=1in, bordercolor=0 0 0, name=p3q1, multiline=true]{}

        \item What was Bob able to decrypt?

            \TextField[width=6in,height=1in, bordercolor=0 0 0, name=p3q2, multiline=true]{}

        \item What was Eve able to encrypt?

            \TextField[width=6in,height=1in, bordercolor=0 0 0, name=p3q3, multiline=true]{}

        \item How private was the message?

            \TextField[width=6in,height=1in, bordercolor=0 0 0, name=p3q4, multiline=true]{}

        \item For what purposes would you use this level of encryption?

            \TextField[width=6in,height=1in, bordercolor=0 0 0, name=p3q5, multiline=true]{}

    \end{enumerate}


    \newpage
    \huge Bonus Exercise
    \normalsize

    As an addition to the manual exercise above, your group will now be guided through how the above steps can be easily automated through a simple python script. The general generation of keys, encoding and decoding of messages, and encrypting and decrypting of messages is provided through an rsa.py file. It is your task to create character files for Bob, Alice, and Eve that will perform the steps you and your group members had to do manually.

    \paragraph{alice.py:} Alice is the simplest of our three characters as all that is required for this portion is to encode, encrypt, and return a message.

    \begin{enumerate}
        \item Import the rsa.py file.

            \TextField[width=6in,height=1in, bordercolor=0 0 0, name=p3q1, multiline=true]{}

        \item Gather the inputs from Bob, aka the public key and the key length.
            
            \TextField[width=6in,height=1in, bordercolor=0 0 0, name=p3q1, multiline=true]{}

        \item Use the functions from the rsa.py file to encode a message.

            \TextField[width=6in,height=1in, bordercolor=0 0 0, name=p3q1, multiline=true]{}

        \item Once you have your encoded message, encrypt the message.

            \TextField[width=6in,height=1in, bordercolor=0 0 0, name=p3q1, multiline=true]{}

        \item Print the encrypted message as a string.

            \TextField[width=6in,height=1in, bordercolor=0 0 0, name=p3q1, multiline=true]{}

    \end{enumerate}

    \paragraph{bob.py:} Bob’s task is slightly more complex as prime inputs will be required and the numpy and sympy libraries will be utilised.

    \begin{enumerate}
        \item Import the rsa.py file, and the numpy and sympy libraries.

            \TextField[width=6in,height=1in, bordercolor=0 0 0, name=p3q1, multiline=true]{}

        \item Create a global boolean variable to keep track of whether prime input will be necessary.

            \TextField[width=6in,height=1in, bordercolor=0 0 0, name=p3q1, multiline=true]{}

        \item If prime input is necessary, then prompt the user to give two primes. If not,  proceed as below.

            \TextField[width=6in,height=1in, bordercolor=0 0 0, name=p3q1, multiline=true]{}

        \item Now prompt the user to enter a lower bound for the amount of characters in the message.

            \TextField[width=6in,height=1in, bordercolor=0 0 0, name=p3q1, multiline=true]{}

        \item Using pow(), generate a lower length, setting 10 as the base, and the length from above as the exponent.

            \TextField[width=6in,height=1in, bordercolor=0 0 0, name=p3q1, multiline=true]{}

        \item To generate our primes, we will use sympy.randprime(), where our input will be lower, and lower*2. Do this twice such that values of p and q are present.

            \TextField[width=6in,height=1in, bordercolor=0 0 0, name=p3q1, multiline=true]{}

        \item To ensure our primes are unique, a small while loop is created such that while our p value is equal to our q value, the q value will be regenerated using the same method as above.

            \TextField[width=6in,height=1in, bordercolor=0 0 0, name=p3q1, multiline=true]{}

        \item Finally, we print our values for p and q as a string, separated by commas.

            \TextField[width=6in,height=1in, bordercolor=0 0 0, name=p3q1, multiline=true]{}

        \item Now we must generate and return these keys to Bob so he may share them with Alice and Eve, to do this we use our rsa.py file, and use generate_keys to encrypt our p and q values. We also pull the key length.

            \TextField[width=6in,height=1in, bordercolor=0 0 0, name=p3q1, multiline=true]{}

        \item Print all three values (the public key, private key, and key length) as strings.

            \TextField[width=6in,height=1in, bordercolor=0 0 0, name=p3q1, multiline=true]{}

        \item Lastly, once Alice has created and encrypted a message, Bob can decode it, thus we create a value to store the message from Alice.

            \TextField[width=6in,height=1in, bordercolor=0 0 0, name=p3q1, multiline=true]{}

        \item Then we use decrypt_message from the rsa.py file, feeding it the above message from Alice, the private key, and the key length.

            \TextField[width=6in,height=1in, bordercolor=0 0 0, name=p3q1, multiline=true]{}

        \item Finally, we use decode_message from the rsa.py file on the above value obtained from decrypt_message to obtain the decrypted message from Alice.

            \TextField[width=6in,height=1in, bordercolor=0 0 0, name=p3q1, multiline=true]{}

    \end{enumerate}

\end{document}
